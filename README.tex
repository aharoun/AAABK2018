%% LyX 2.2.2 created this file.  For more info, see http://www.lyx.org/.
%% Do not edit unless you really know what you are doing.
\documentclass[english]{article}
\usepackage[T1]{fontenc}
\usepackage[latin9]{inputenc}
\usepackage{geometry}
\geometry{verbose,tmargin=2cm,bmargin=3cm,lmargin=4cm,rmargin=4cm}
\setlength{\parindent}{0bp}
\usepackage{xcolor}
\usepackage{float}
\usepackage{booktabs}

\makeatletter

%%%%%%%%%%%%%%%%%%%%%%%%%%%%%% LyX specific LaTeX commands.
%% Because html converters don't know tabularnewline
\providecommand{\tabularnewline}{\\}

\makeatother

\usepackage{babel}
\begin{document}

\title{Code for ``Innovation, Reallocation and Growth'' }

\author{Daron Acemoglu \ \ \  Ufuk Akcigit \ \ \  Harun Alp \\
Nicholas Bloom \ \ \  William Kerr}

\maketitle
This file contains an overview of the code for replicating the results.
It is implemented in \texttt{MATLAB}. There are 5 folders which refer
to different versions of the model. Each folder includes a \texttt{\textcolor{magenta}{MASTER.m}}\texttt{
}file which produces the results for the corresponding model. Final
output is compiled, as they appear in the paper, under the subfolder
\texttt{\textbf{\textcolor{purple}{output}}}. A more detailed description
of the code is given below. 

\bigskip{}


\subsubsection*{Parameters}

There are 13 parameters of the model, defined in the code as follows:

\begin{table}[H]

\centering{}%
\begin{tabular}{rl}
\toprule 
\textbf{\textit{lam:}} & Innovation step size\tabularnewline
\textbf{\textit{psi:}} & Exogenous destruction rate\tabularnewline
\textbf{\textit{nu:}} & Transition rate from high-type to low-type\tabularnewline
\textbf{\textit{alpha:}} & Probability of being high-type entrant\tabularnewline
\textbf{\textit{phi:}} & Fixed cost of operation\tabularnewline
\textbf{\textit{theta\_l:}} & Innovative capacity of low-type firms\tabularnewline
\textbf{\textit{theta\_h:}} & Innovative capacity of high-type firms\tabularnewline
\textbf{\textit{theta\_e:}} & Innovative capacity of entrants\tabularnewline
\textbf{\textit{eps:}} & CES parameter\tabularnewline
\textbf{\textit{disc:}} & Discount rate\tabularnewline
\textbf{\textit{gamma:}} & Innovation elasticity for incumbents\tabularnewline
\textbf{\textit{sigma:}} & Inverse of the intertemporal elasticity of substitution\tabularnewline
\textbf{\textit{Ls:}} & Measure of high-skilled workers\tabularnewline
\bottomrule
\end{tabular}
\end{table}

Parameters are read from a text file under the subfolder \texttt{\textbf{\textcolor{purple}{params}}}.
Global structure \textbf{\textit{p}} keeps parameter values currently
being used, as well as policy values.

\subsubsection*{Equilibrium solver}

Equilibrium of the model can be characterized as a solution to a system
of 6 equations in 6 unknowns: \textbf{(i)} wage rate {[}\textbf{\textit{ws}}{]},
\textbf{(ii)} mass of active product lines owned by low type firms
{[}\textbf{\textit{cactiv(1)}}{]}, \textbf{(iii)} mass of active product
lines owned by high type firms {[}\textbf{\textit{cactiv(2)}}{]},
\textbf{(iv)} expected value to a low type firm of a newly innovated
product line {[}\textbf{\textit{eyq(1)}}{]}, \textbf{(v)} expected
value to a high type firm of a newly innovated product line {[}\textbf{\textit{eyq(2)}}{]}
and \textbf{(vi)} average quality {[}\textbf{\textit{qbar}}{]}. Global
structure \textbf{\textit{eq}} stores all the relevant equilibrium
objects and passes them over to different routines. Initial guess
for the solution is logged under subfolder \texttt{\textbf{\textcolor{purple}{eqvars}}}. 

\bigskip{}

The following are the most important files for solving the equilibrium:
\begin{itemize}
\item \texttt{\textcolor{magenta}{initalg.m}}\textbf{:} It creates the global
structure \textbf{\textit{alg }}that contains tuning parameters, binary
switches and file names. 
\item \texttt{\textcolor{magenta}{solver.m}}\textbf{:} It loads in model
parameters, and initial guess for equilibrium variables and run \texttt{\textcolor{magenta}{fsolve}}
on the equilibrium function in \texttt{\textcolor{magenta}{eqfunc.m}}. 
\item \texttt{\textcolor{magenta}{eqfunc.m}}\textbf{:} It takes in guessed
equilibrium variables and returns the equation errors which are calculated
based on the following routines:
\begin{itemize}
\item \texttt{\textcolor{magenta}{innovation.m}}\textbf{:} It finds innovation
rates (\textbf{\textit{x}}) and the minimum quality of a product line
for each type of firms (\textbf{\textit{qmin}}).
\item \texttt{\textcolor{magenta}{qualityDist.m}}\textbf{:} It solves the
quality distribution and calculates the updated mass of active product
lines.
\item \texttt{\textcolor{magenta}{qbarActive.m}}\textbf{:} It computes the
average quality of active product lines.
\item \texttt{\textcolor{magenta}{calcey.m}}\textbf{:} It uses quality distribution
to find updated expected product line values.
\item \texttt{\textcolor{magenta}{labordem.m}}\textbf{:} Using innovation
rates and wage rate, it finds labor demand.
\end{itemize}
\end{itemize}

\subsubsection*{Estimation}

Estimation routine is implemented in \texttt{\textcolor{magenta}{smm.m}}.
This routine searches over the parameter space to minimize the distance
between simulated and data moments. The objective function for the
estimation is \texttt{\textcolor{magenta}{smmobj.m}} which takes the
proposed parameter values, solves the model, simulates a panel of
firms and calculates various moments of interest. Firm simulation
and moment calculations are done in \texttt{\textcolor{magenta}{compMoments.m}},
which itself calls the mex file \texttt{\textcolor{magenta}{firmsim.mexmaci64}}.
This file is compiled from the source file \texttt{\textcolor{magenta}{firmsim.cpp}}
(in \texttt{C++}, located under subfolder \texttt{\textbf{\textcolor{purple}{mex}}})
for a Mac (64-bit) machine. The source file is needed to be recompiled
for any other platform. To speed up the simulation, parallelization
option is available, which can be controlled by the parameters in
\texttt{\textcolor{magenta}{initalg.m}}. Finally, \texttt{\textcolor{magenta}{bootstrapSD.m}}
computes standard errors for the estimated parameters based on bootstrap. 

\subsubsection*{Optimal Policies}

The social planners problem is solved in \texttt{\textcolor{magenta}{socplan\_opt.m}},
which uses \texttt{\textcolor{magenta}{socplan\_solver.m}} and \texttt{\textcolor{magenta}{socplan\_eqfunc.m}}.
For the baseline model, social planner's choice variables are: (i)
minimum quality for low type firms {[}\textbf{\textit{qmin(1)}}{]},
(ii) minimum quality for high type firms {[}\textbf{\textit{qmin(2)}}{]},
(iii) innovation rate for low type firms {[}\textbf{\textit{x(1)}}{]},
and (iv) innovation rate for high type firms {[}\textbf{\textit{x(2)}}{]}.\textcolor{magenta}{{}
}\texttt{\textcolor{magenta}{run\_pols.m}} finds subsidy rate for
different policies that corresponds to 1\% of GDP. Finally, optimal
subsidy policy is run through \texttt{\textcolor{magenta}{policy\_opt.m}}.
It searches over subsidy rates to maximize welfare.
\end{document}
